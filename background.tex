\section{Safer smart contracts}
\label{background}

% Electronic contracts have early beginnings in the AI and agent community where they are used as a basis for interaction~\cite{Smith1980}. Moreover, electronic contracts are discussed and used for automating or encoding traditional contracts in organisational and business contexts~\cite{Hvitved2010}. Specifically, Ricardian contracts are oft-cited as they introduce encoding natural language contracts into electronic contracts that can be executed by computer systems~\cite{Grigg2004}. Later, electronic contracts are applied to enforce agreements between mutually untrusted entities~\cite{Szabo1997}.


% If everyone trusted each other, contracts would not be a necessity.
% However, a significant amount of interaction is conducted between untrusted or semi-trusted parties.


% \dha{Rename to history of smart contracts?}
% Contracts are a requirement resulting from an inherent lack of trust between parties.
%\subsubsection{Electronic contracts}
%Electronic contracts have been considered as a way to encode legal contracts (e.g. Ricardian contracts~\cite{Grigg2004}) in a Domain Specific Language (DSL). 
%They can be processed digitally enabling automated business processes.
%Encoding contracts is surveyed in~\cite{Hvitved2010}. 
%The author list 13 different requirements that a language needs to fulfil.
%He found that languages can be roughly categorised into four different formalisms including deontic logic, event-condition-action, action-trace, and others.
%Most of these languages do not have formal semantics except for~\cite{Andersen2006,Kyas2008,Xu2004}.
%Rather than enforcing contracts through a consensus protocol, those systems monitor compliance of participants in the form of norms (permissions, obligations, and prohibitions), e.g.\~\cite{Kyas2008}.
%\subsubsection{Smart contracts}
Smart contracts execute agreements on a distributed ledger. %like Bitcoin or Ethereum~\cite{Nakamoto2008,Buterin2013}.
%The ledger's consensus protocol ensures correct execution of the smart contract, whereby a majority or super-majority agrees on the accepted result.
%Consensus protocols enable \emph{distrusting parties} to create contracts and interact.
However, no single definition of a smart contract exists.
Information and definitions are scattered in academia and various communities (e.g.\ Bitcoin and Ethereum Improvement Proposals).
Further, the capabilities of smart contracts range from minimal (e.g.\ Bitcoin), to restrictive (e.g.\ Scilla~\cite{Sergey2018}), to Turing-complete instruction sets (e.g.\ Ethereum). 
We provide a brief overview of the technical components of a smart contract and the underlying execution environment.
%Whether or not smart contracts have legal implications or not is also widely discussed.
% Likewise, parties that have a form of \emph{trusted relationship} can utilise smart contracts on a ledger with stronger trust assumptions.
% This can be beneficial since, e.g. transaction rates can be higher and fees (if any) lower.
%Moreover, there is a caveat with distributed ledgers and consensus protocols.
%Results of contracts need to be fully deterministic so that each (honest) actor in the consensus protocol reaches a single common output with the same set of inputs.

%\dha{No unique definition to smart contracts - Ethereum and Bitcoin understanding entirely different. Many information is scattered in developments, EIP/BIP, academia.}

\subsection{Technical overview}
%\dha{Explain the smart contract ``stack''}
%Smart contracts need a low-level language that allows deterministic execution. 
%A high-level language can make it easier for developers to create new contracts and reason about existing contracts.
%In software development, having different sets of languages is a conventional process.
The purpose of the distributed ledger plays a vital role in the design of the language and execution environment.
Languages and execution environments that are permissive can be used to implement a wide range of applications. 
Contrarily, special purpose ledgers can restrict language and execution features to provide strong security guarantees.

\subsubsection{Distributed ledger}
We can distinguish between two different ways of storing contracts in distributed ledgers including Unspent Transaction Output (UTXO) or account-based.
Bitcoin is an example for a UTXO model~\cite{Nakamoto2008,Covaci2018}, where contracts are stored as part of a transaction. %in the \texttt{scriptSig} and \texttt{scriptPubKey} of a transaction. 
Hence, contracts implement conditions of a single transaction or need to be ``chained'' over multiple ones.
Account-based ledgers store contracts at a specific address.
Contracts have a local state, and their functions can be called multiple times.
State-changing functions are invoked by sending a transaction to a contract function exposed via its Application Binary Interface (ABI)~\cite{Wood2014,Sergey2018,OCamlProSAS2018}.
Contracts can typically access a global state to receive information stored in the ledger such as block timestamps, block hashes, or transaction data. 

\subsubsection{Smart contract}
Smart contracts are implemented by a low-level language.
They are executed on a distributed virtual machine (VM) by multiple nodes in the distributed ledger.
The outcome of a contract needs to be deterministic to reach consensus over its outcome.
The low-level language ultimately determines how ``restrictive'' programs and protocols are on its ledger.
While it would be sufficient to only provide a low-level language, it is often beneficial in software development to have languages with different abstraction levels and compilers in between.
Apart from the low-level language, there are two more levels of languages for smart contracts.
A high-level language can make it easier for developers to create new contracts and reason about existing contracts.
Multiple high-level languages can exist in parallel to be executed on the same ledger. 
They offer the highest level of abstraction.
%Examples are Solidity~\cite{Ethereum2018Solidity} and Liquidity~\cite{OCamlProSAS2018}.
Intermediary representations (IR) are in between low-level and high-level languages. 
An IR can be a common target for multiple high-level languages.
IRs can be used to reason about smart contract properties (like safety or liveness) and build a basis for optimising code. 
High-level languages in return can profit from these methods and optimisations without having to implement them directly into their compilers.
%Examples include Simplicity~\cite{OConnor2017} and Scilla~\cite{Sergey2018}.

% However, this can introduce security issues when dependencies on the global state are introduced within the contract that can be influenced by parties outside of the contract (typically miners). Other issues revolve around calling into other, possibly faulty or unpredictable, contracts.

%\dha{More detail on the ledger? Bitcoin uses to determine if a tx can be spent. Generally, ledger consists of transactions and scripts, consensus protocol, and network protocol.}

%\dha{Explain verification efforts. Combination of smart contracts (the logic to be verified) and the ledger environment (the runtime environment).}

\subsection{Security properties}
The general properties of safety and liveness apply to smart contracts as well~\cite{Sergey2018}.
Safety refers to satisfying specific correctness properties during any state of a contract.
Liveness describes that certain events may eventually occur.
Analysis of vulnerabilities violating security properties with a focus on the EVM is presented in~\cite{Luu2016} and~\cite{Atzei2017}.
\citeauthor{Grishchenko2018} extend this work by introducing a formal definition of general security properties for smart contracts with a focus on account-based distributed ledgers~\cite{Grishchenko2018}.
Further, privacy of smart contracts is a concern~\cite{Kosba2015,Bowe2018,Covaci2018}.
Last, smart contracts can have deadlocks due to metering restrictions and bugs in the implementation~\cite{Grech2018}.
Hence, we add a liveness property as well.
We note that the listed properties might already be fulfilled due to the implementation details by the ledger. 
For example, UTXO ledgers are atomic by design while this property needs to be implemented by smart contracts in account-based ledgers.

\subsubsection{Call integrity}
A contracts state may depend on an external call. Specifically, a contract can (i) execute code of an external contract within its functions, (ii) call another contract's function and wait for its returned value, (iii) call another contract that changes the global state or re-enters the calling contract, or (iv) relies on an external oracle that updates the state of the contract.
The contract control flow should not be influenced by an adversary contract or oracle.

\subsubsection{Atomicity}
Functions should be executed entirely or the state should be reverted, except when specifically allowed during exception handling.
Contracts might execute only a part of the transaction due to calls to other contracts during function execution, stack limits, or enforced stops due to metering limitations.
Partial function execution might leave the contract open to malicious acts.

\subsubsection{Independence}
Transactions change the state of a contract. Miners and other parties can control or influence parameters in contracts or transactions. For example, a contract might use a timestamp-dependent function, which can in certain ranges be influenced by miners. Also, the ordering of transactions can be influenced by miners or others paying higher fees. Hence, contracts should ideally be independent of the global state or parameters that can be influenced externally.

\subsubsection{Privacy}
Transactions are hidden from the public except when explicitly and voluntarily disclosed by the involved parties.
Smart contracts typically inherit the privacy assumptions from their underlying chain.
However, advanced techniques such as zero-knowledge proofs can be employed to realise private smart contracts even on public non-privacy-preserving ledgers.

\subsubsection{Liveness}
A contracts state should not be locked, if not explicitly defined.
For example, a contract might be forever stuck if it needs to loop through a too large array causing an \emph{out-of-gas} exception.


\subsubsection{Runtime correctness}
The properties above apply to general contracts. However, each contract serves a distinct purpose. Properties particular to the expected behaviour of a contract need to be defined to ensure the runtime correctness of the contract.

%\subsection{Formalisation}
%$\Phi$ specification
%$\mathcal{M}$ model
%$\mathcal{M} \models \Phi$ satisfies
%
%$S$ state
%$S \rightarrow S'$ state transition
%$T \in \mathcal{T}$ transactions
%$S' \leftarrow F_{C}(S, T)$

%\dha{What about a short review about security issues? Desired properties are safety and liveness.}

% \subsection{Contract verification}

%\subsection{Survey objective}
%Smart contracts potentially handle large amounts of money.
%Hence, keeping them safe requires efforts on different levels.
%We summarise the efforts taken to create safe smart contracts by answering the following research question.
%
%\paragraph{\textbf{RQ:}} How can languages and verification methods contribute to safer smart contracts?
%
%
%\paragraph{} In detail, we present relevant previous work according to languages and verification tools.
%High-level languages can encourage secure programming principles that reduce the number of bugs a programmer introduces.
%IR and low-level languages can be designed to simplify program analysis and verification.
%Verification tools can analyse specific properties or full contracts by using representative models or proof methods.
%% Their impact can be promoted by full coverage and ease of use.
%Last, verified compilers are a practice to allow reasoning of programs at a higher level and safe execution at a lower level.
% Design by contract~\cite{Meyer1992}