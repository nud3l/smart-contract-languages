\documentclass{article}
\usepackage[utf8]{inputenc}



\usepackage[backend=biber,style=numeric,citestyle=ieee,doi=false,isbn=false,arxiv=false]{biblatex}

\AtEveryBibitem{\clearfield{publisher}}
\AtEveryBibitem{\clearfield{editor}}
% Fix Mendely URL stuff 1
\AtEveryBibitem{\ifentrytype{article}{\clearfield{url}}{}}
\AtEveryBibitem{\ifentrytype{inproceedings}{\clearfield{url}}{}}
\AtEveryBibitem{\ifentrytype{book}{\clearfield{url}}{}}
\AtEveryBibitem{\ifentrytype{incollection}{\clearfield{url}}{}}
% Fix Mendeley URL stuff 2
\DeclareSourcemap{
    \maps{
        \map{ % Replaces '{\_}', '{_}' or '\_' with just '_'
            \step[fieldsource=url,
                  match=\regexp{\{\\\_\}|\{\_\}|\\\_},
                  replace=\regexp{\_}]
        }
        \map{ % Replaces '{'$\sim$'}', '$\sim$' or '{~}' with just '~'
            \step[fieldsource=url,
                  match=\regexp{\{\$\\sim\$\}|\{\~\}|\$\\sim\$},
                  replace=\regexp{\~}]
        }
    }
}

% \addbibresource{ref.bib} % Overleaf
\addbibresource{../../bib/library.bib} % local


\title{Towards Safer Smart Contract: an Overview on Languages and Verification}
\date{}
\author{
Dominik Harz\\ \texttt{d.harz@imperial.ac.uk} \and
William Knottenbelt\\ \texttt{wjk@imperial.ac.uk}
}


\begin{document}
\maketitle

\begin{abstract}
Since their inception, smart contract
We present a systematic review of existing smart contract languages
\end{abstract}

\section{Introduction}
The idea of contracts between autonomous parties goes back to autonomous agents using a network of agents to solve tasks dsitributed and based on individual contracts as presented in the contract net protocol (CNP) \cite{Smith1980,FIPA2002}.
The idea was further elaborated, with a focus on minimising or rather excluding the need for trust in either party, by Szabo coining the term ``smart contract'' \cite{Szabo1997}.
Significant thought has been directed towards creating languages and framework for electronic contracts even before the inception of distributed ledgers, for example, \cite{Andersen2006,Kyas2008,Xu2004}.
However, distributed ledgers are the ones facilitating wide-spread adoption of electronic contracts.


Smart contracts based on distributed ledgers combine two curious properties: anyone can freely create such contracts for the whole world to interact with, while each line of code (LoC) might affect significant amount of currency (in the range of millions of USD per LoC).
These contracts allow economic interactions between different parties without the need to trust one another.
This includes contracts for financial services, notaries, games, wallets, or libraries \cite{Bartoletti2017}.
Further, smart contracts are the enabler of protocols build on top of distribued legers including, for example, Lightning \cite{Poon2016}, Plasma \cite{Poon2017}, Polkadot \cite{Wood2017}, and TrueBit \cite{Teutsch2017}.
However, security incidents caused by software bugs has lead to severe losses as in the infamous The DAO incident \cite{Daian2016}, and Parity multi-sig vulnerabilities \cite{Breidenbach2017Parity,ParityTech2017}. 


Major efforts are taken to prevent such future incidents including:
\begin{itemize}
\item Introducing new high-level programming languages to encourage safe practices, for example  \cite{Hirai2018Bamboo,Ethereum2018Vyper,Schrans2018}.
\item Developing safe languages for distributed virtual machines, for example \cite{Sergey2018,DynamicLedgerSolutions2017,Popejoy2017,Kasampalis2018}.
\item Analysing source code by symbolic modelling and execution, for example \cite{Luu2016,Tsankov2017,Kalra2018,Albert2018}.
\item Formal semantics and verification, for example \cite{Bhargavan2016,Hildenbrandt2017,Hirai2017}.
\end{itemize}

\textbf{Contribution} The amount of new languages, approaches for verification, and applicability of verification methods becomes quickly opaque. Due to the practical impact of these approaches to real-world smart contracts, we present a literature survey on current languages and verification efforts.
\begin{itemize}
\item We contribute an overview of contract languages including a classification of language paradigm, key security features, and state of implementation\footnote{As of September 2018}.
\item We describe different efforts to verify software including symbolic analysis and formal verification. This includes an analysis of applicability of different techniques, limitations, and current availability of methods.
\end{itemize}

\textbf{Structure}: The remainder of our article is structured as follows. Section \ref{background} introduces the background of contracts and languages to express them. In section \ref{method} we describe how we conducted our survey. We present an overview and a classification of languages in section \ref{languages}. Similarly, verification approaches are examined in section \ref{verification}. We conclude in section \ref{conclusion}.


\section{Contracts and ledgers}
\label{background}

High-level languages

distributed virtual machine

Expression of contracts

Bitcoin \cite{Nakamoto2008}
Bitcoin Script \cite{BitcoinWiki2018Script}

Ethereum \cite{Buterin2013,Wood2014}

existing reviews:

Smart contracts for distributed ledger technology with focus on Bitcoin Script, Ethereum, Nxt. Mentionds interactions between DLT layer (virtual machine) and the language interacting with the DLT layer (like Solidity)
\cite{Seijas2017}


An Empirical Analysis of Smart Contracts: Platforms, Applications, and Design Patterns
Focus on Bitcoin and Ethereum
Introduce taxonomy of smart contracts according to usage categories
Design patterns in Ethereum smart contracts
\cite{Bartoletti2017}

Survey of Formal Languages for Contracts with a list of thirteen requirements for contract formalisaiton.
(R1) modelling ofcontract participants; (R2) parametrized contract templates; (R3) (conditional) commitments, i.e., obligations, permissions, and prohibitions; (R4) abso- lute and relative temporal constraints; (R5) history-sensitive commitments; and (R6) basic arithmetic
(R7) contrary-to-duty (reparation clauses); (R8) potentially infinite and repetitive contracts; (R9) compositionality; (R10) deterministic contract execution (run-time monitoring); (R11) blame as- signment; (R12) the isomorphism principle; and (R13) subject to analysis
 \cite{Hvitved2010}.
 
(deontic) logic based formalisms
event-condition-action based formalisms enforce
action/trace based formalisms
other formalisms

Few have formal semantics \cite{Andersen2006} \cite{Kyas2008} \cite{Xu2004}

None cover all the requirements


Scilla includes a survey of 13 different smart contract languages in section 5 \cite{Sergey2018}

Securify has a comparison to related work in section 8 \cite{Tsankov2017}



\section{Method}
\label{method}

imperative
- object-oriented
- procedural
declarative
- functional
- logic
symbolic



\section{Contract languages}
\label{languages}


Logic
normative system with rights and obligations based on Prolog \cite{Michael2010}


Software engineering
Design by contract \cite{Meyer1992}

Business Contract Language \cite{Neal.2003}

Domain Specific Languages
QoS Contract langauge formal specification and verification with Maude \cite{Braga2009} Maude \cite{Clavel2007}
Business Contract Language formal analysis \cite{Governatori2006}


Smart contract templates
Usage of Ricardian contract triple design pattern \cite{Clack2016}


Event calculus
XML formalisation of Event Calculus for tracking state of smart contracts \cite{Farrell2004}

\textbf{Higher level}
Solidity \cite{Ethereum2018Solidity}
Vyper \cite{Ethereum2018Vyper}
Bamboo \cite{Hirai2018Bamboo}
Flint \cite{Schrans2018}


Functional
DAML \cite{Shaul2018,Meier2018,Lippmeier2018,Huschenbett2018,Bernauer2018,Maric2018,Bleikertz2018,Lochbihler2018,Pilav2018}
Simplicity \cite{OConnor2017} with formal semantics in Coq
Scilla \cite{Sergey2018}
Pyramid Scheme \cite{Burge2018}

Rootstock as an extension or rather pegged sidechain to allow EVM contracts with Bitcoin \cite{Lerner2015}

VM level
Stack-based
Bitcoin script \cite{BitcoinWiki2018Script}
EVM \cite{Wood2014}
EWASM \cite{Wanderer2015,EthereumFoundation2018ewasm}

Register-based
IELE \cite{Kasampalis2018}


Liquidity \cite{OCamlProSAS2018}

Scilla \cite{Sergey2018}
Michelson (Tezos) \cite{DynamicLedgerSolutions2017}
Pact \cite{Popejoy2017}
Rholang \cite{Meredith2018}

Container
Hyperledger with Go, Java, Node.js \cite{Cachin2016}

\section{Validation and Verification}
\label{verification}
\textbf{Formal methods}

F*
Formal Verification first paper \cite{Bhargavan2016}
EVM semantics in F* \cite{Grishchenko2018}

K framework \cite{Rosu2007} 
KEVM \cite{Hildenbrandt2017}
Language independent \cite{Chen2018}
ERC20 etc \cite{Park2018}

Lem based \cite{Mulligan2014}
EVM definition in Lem \cite{Hirai2017}
Towards verifying contracts \cite{Amani2018}
Overview of work by Ethereum Foundation mostly \citeauthor{Hirai2018} \cite{Hirai2018}


\textbf{Model checking}
Securify \cite{Tsankov2017}
Oyente \cite{Luu2016}
Mythril \cite{Mueller2018}
Maian \cite{Nikolic2018}

Zeus \cite{Kalra2018}
Detection of Effectively Callback Free Objects \cite{Grossman2017}



\textbf{Proof carrying code}
NECTAR as an extension to Bitcoin's UTXO model and Script set \cite{Covaci2018}

\textbf{Game theoretic approaches}
gmae theory and markov devision process (MDP) model \cite{Bigi2015}



\section{Conclusion}
\label{conclusion}
Hydra \cite{Breidenbach2018}


\printbibliography

\end{document}
