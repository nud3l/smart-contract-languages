\documentclass{article}
\usepackage[utf8]{inputenc}
\title{Towards Safer Smart Contract: a Survey}
\date{}
\author{Dominik Harz \\
IC3RE\\
d.harz@imperial.ac.uk}


\begin{document}
\maketitle

\begin{abstract}
Since their inception, smart contract
We present a systematic review of existing smart contract languages
\end{abstract}

\section{Introduction}
Smart contracts based on distributed ledgers combine two peculiar properties: anyone can freely create such contracts for the whole world to access, while each line of code might affect millions.
These contracts are successful in their own regard:
Ethereum, the largest platform for such contracts, has no X contracts controlling a combined value of X million USD.
However, security incidents caused by software bugs has lead to severe losses as in the DAO, Parity (1 and 2), and others.
A majority of the current work concentrates on creating safer smart contracts.


This survey takes on the verification perspective of these efforts and presents a survey of current work.

\section{}

\section{Electronic contracts}

Bitcoin \cite{Nakamoto2008}
Bitcoin Script \cite{BitcoinWiki2018Script}

Ethereum \cite{Buterin2013,Wood2014}

existing reviews:

Smart contracts for distributed ledger technology with focus on Bitcoin Script, Ethereum, Nxt. Mentionds interactions between DLT layer (virtual machine) and the language interacting with the DLT layer (like Solidity)
\cite{Seijas2017}


An Empirical Analysis of Smart Contracts: Platforms, Applications, and Design Patterns
Focus on Bitcoin and Ethereum
Introduce taxonomy of smart contracts according to usage categories
Design patterns in Ethereum smart contracts
\cite{Bartoletti2017}

Survey of Formal Languages for Contracts with a list of thirteen requirements for contract formalisaiton.
(R1) modelling ofcontract participants; (R2) parametrized contract templates; (R3) (conditional) commitments, i.e., obligations, permissions, and prohibitions; (R4) abso- lute and relative temporal constraints; (R5) history-sensitive commitments; and (R6) basic arithmetic
(R7) contrary-to-duty (reparation clauses); (R8) potentially infinite and repetitive contracts; (R9) compositionality; (R10) deterministic contract execution (run-time monitoring); (R11) blame as- signment; (R12) the isomorphism principle; and (R13) subject to analysis
 \cite{Hvitved2010}.
 
(deontic) logic based formalisms
event-condition-action based formalisms enforce
action/trace based formalisms
other formalisms

Few have formal semantics \cite{Andersen2006} \cite{Kyas2008} \cite{Xu2004}

None cover all the requirements


Scilla includes a survey of 13 different smart contract languages in section 5 \cite{Sergey2018}

Securify has a comparison to related work in section 8 \cite{Tsankov2017}



\section{Method}



\section{Contract languages}

agent systems
CNP \cite{Smith1980}
FIPA CNP \cite{FIPA2002}

Logic
normative system with rights and obligations based on Prolog \cite{Michael2010}


Software engineering
Design by contract \cite{Meyer1992}

Business Contract Language \cite{Neal.2003}

Domain Specific Languages
QoS Contract langauge formal specification and verification with Maude \cite{Braga2009}
Business Contract Language formal analysis \cite{Governatori2006}


Smart contract templates
Usage of Ricardian contract triple design pattern \cite{Clack2016}


Event calculus
XML formalisation of Event Calculus for tracking state of smart contracts \cite{Farrell2004}

\textbf{Higher level}
Solidity \cite{Ethereum2018Solidity}
Vyper \cite{Ethereum2018Vyper}
Bamboo \cite{Hirai2018Bamboo}




Functional
DAML \cite{Shaul2018,Meier2018,Lippmeier2018,Huschenbett2018,Bernauer2018,Maric2018,Bleikertz2018,Lochbihler2018,Pilav2018}
Simplicity \cite{OConnor2017} with formal semantics in Coq
Scilla \cite{Sergey2018}
Pyramid Scheme \cite{Burge2018}

Rootstock as an extension or rather pegged sidechain to allow EVM contracts with Bitcoin \cite{Lerner2015}

VM level
Stack-based
Bitcoin script \cite{BitcoinWiki2018Script}
EVM \cite{Wood2014}
EWASM \cite{Wanderer2015,EthereumFoundation2018ewasm}

Register-based
IELE \cite{Kasampalis2018}

Not clear
Scilla \cite{Sergey2018}
Michelson (Tezos) \cite{DynamicLedgerSolutions2017}
Pact \cite{Popejoy2017}

Container
Hyperledger with Go, Java, Node.js \cite{Cachin2016}

\section{Validation and Verification}
\textbf{Formal verification}

F*
Formal Verification first paper \cite{Bhargavan2016}

K framework \cite{Rosu2007} 
KEVM \cite{Hildenbrandt2017}
Language independent \cite{Chen2018}
ERC20 etc \cite{Park2018}

Lem based
Towards verifying contracts \cite{Amani2018}




\textbf{Symbolic analysis}
Securify \cite{Tsankov2017}




\textbf{Proof carrying code}
NECTAR as an extension to Bitcoin's UTXO model and Script set \cite{Covaci2018}



\section{Related work}
Hydra \cite{Breidenbach2018}



\bibliographystyle{ieeetr}
% \bibliography{ref.bib} % Overleaf
\bibliography{../../bib/library.bib} % local

\end{document}
