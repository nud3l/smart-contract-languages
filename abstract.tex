\begin{abstract}
% With a market capitalisation of over USD 205 billion in just under ten years, public distributed ledgers have experienced significant adoption.
Smart contracts are an integral part of blockchains.
They are programs which enable distrusting parties to conduct token sales, maintain multi-signature wallets, and engineer entirely new protocols on top of their root blockchain.
% These programs allow distrusting parties to enter agreements that are executed autonomously.
However, implementation issues in smart contracts caused severe losses, for example, in The DAO and Parity wallets.
Security remains a challenge as known vulnerabilities are still found in smart contracts deployed in production networks.
Significant efforts are taken to improve their security by introducing new programming languages and advance verification methods.
We provide a survey of those efforts in two parts.
First, we introduce an overview of smart contract languages focusing on security features.
To that end, we discuss these languages under the aspects of paradigm, type, instruction set, semantics, and metering.
Second, we introduce verification tools and methods for smart contract and distributed ledgers. 
Accordingly, we examine their verification approach, level of automation, coverage, and supported languages.
Last, we present future research directions including compositional and holistic specifications, specification languages, automated verification, proof-carrying code, and economic properties.
%\dha{Update future research.}
\end{abstract}