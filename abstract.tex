\begin{abstract}
% With a market capitalisation of over USD 205 billion in just under ten years, public distributed ledgers have experienced significant adoption.
Smart contracts are an integral part of blockchains.
They are programs which enable distrusting parties to conduct token sales, creating multi-signature wallets, and engineering entirely new protocols on top of their root blockchain.
% These programs allow distrusting parties to enter agreements that are executed autonomously.
However, implementation issues in smart contracts caused severe losses in for example The DAO and Parity wallets.
Security remains a challenge as known vulnerabilities are still found in smart contracts deployed in production networks.
Significant efforts are taken to improve their security by introducing new programming languages and advance verification methods.
We provide a survey of those efforts in two parts.
First, we introduce several smart contract languages focusing on security features.
To that end, we present an overview concerning paradigm, type, instruction set, semantics, and metering.
Second, we examine verification tools and methods for smart contract and distributed ledgers. 
Accordingly, we introduce their verification approach, level of automation, coverage, and supported languages.
Last, we present future research directions including formal semantics, verified compilers, and automated verification.
\dha{Update future research.}
\end{abstract}