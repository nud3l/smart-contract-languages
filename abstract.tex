\begin{abstract}
Smart contracts are an integral part of blockchains.
They are programs which enable distrusting parties to interact with each other.
Creators of smart contracts can engineer protocols for micro-payments, verified computations, or cross-chain communication.
Further, these programs are the enabler of new economic phenomenons such as token-based funding and digital collectables.
However, implementation issues in smart contracts caused severe losses, for example, in The DAO and Parity wallets.
Security remains a challenge despite significant attempts to create safer languages, raise awareness, and better verification tools.
Significant efforts are taken to improve smart contract security by introducing and improving programming languages as well as advancing verification methods.
We provide a survey of those efforts in two parts.
First, we introduce an overview of smart contract languages focusing on security features.
To that end, we discuss these languages under the aspects of paradigm, type, instruction set, semantics, and metering.
Second, we introduce verification tools and methods for smart contract and distributed ledgers. 
Accordingly, we examine their verification approach, level of automation, coverage, and supported languages.
Last, we present future research directions including compositional and holistic specifications, specification languages, automated verification, proof-carrying code, and economic properties.
\end{abstract}