\section{Introduction}
The idea of contracts between independent parties goes back to autonomous agents using a network of agents to solve tasks distributed and based on individual contracts as presented in the contract net protocol (CNP) \cite{Smith1980}.
The idea was further elaborated, with a focus on minimising and ideally excluding the need for trust in either party, by Szabo coining the term ``smart contract'' \cite{Szabo1997}.
Significant work has been directed towards creating languages and framework for electronic contracts even before the inception of distributed ledgers, for example, \cite{Andersen2006,Kyas2008,Xu2004}.
% However, distributed ledgers are the ones facilitating the wide-spread adoption of electronic contracts.


Smart contracts based on distributed ledgers combine two unique properties: anyone can freely create such contracts for the whole world to interact with, while each line of code (LoC) might affect a significant amount of currency (in the range of millions of USD per LoC).
These contracts allow economic interactions between different parties without the need to trust one another.
This includes contracts for financial services, notaries, games, wallets, or libraries \cite{Bartoletti2017}.
Further, smart contracts are the enabler of protocols build on top of distribued legers including, for example, Lightning \cite{Poon2016}, Plasma \cite{Poon2017}, Polkadot \cite{Wood2017}, and TrueBit \cite{Teutsch2017}.
However, security incidents caused by software bugs has lead to severe losses as in the infamous The DAO incident \cite{Daian2016}, and Parity multi-sig vulnerabilities \cite{Breidenbach2017Parity,ParityTech2017}. 


Significant efforts are taken to prevent such future incidents. 
High-level programming languages are introduced to encourage safe programming practices, for example  \cite{Hirai2018Bamboo,Ethereum2018Vyper,Schrans2018}.
Languages for distributed virtual machines that allow for easy verification are realised, for example \cite{Sergey2018,DynamicLedgerSolutions2017,Popejoy2017,Kasampalis2018}.
Tools for analysing source code by symbolic modelling and execution, for example \cite{Luu2016,Tsankov2017,Kalra2018,Albert2018} as well as formal semantics and verification, for example \cite{Bhargavan2016,Hildenbrandt2017,Hirai2017}, are developed.

\paragraph{Contribution:} The amount of new languages, approaches for verification, and applicability of verification methods becomes quickly opaque. Due to the practical impact of these approaches to real-world smart contracts, we present a literature survey on current languages and verification efforts.
We contribute an overview of contract languages including a classification of language paradigm, essential security features, and state of implementation\footnote{As of September 2018.}.
Further, we describe different efforts to verify software including model- and proof-based methods. This includes an analysis of the applicability of different techniques, limitations, and current availability of methods.

\paragraph{Structure:} The remainder of our article is structured as follows. Section \ref{background} introduces the background of contracts and languages to express them. We present an overview and a classification of languages in section \ref{languages}. Similarly, verification approaches are examined in section \ref{verification}. Our results are discussed including related work in section \ref{discuss}. We conclude in section \ref{conclusion}.